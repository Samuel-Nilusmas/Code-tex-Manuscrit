%% PRÉAMBULE : PACKAGES début
\usepackage[dvipsnames]{xcolor}
\usepackage[a4paper,textwidth=15cm]{geometry}
\usepackage{setspace}
\usepackage{fancyhdr}
\usepackage[Bjornstrup]{fncychap}
%\usepackage[title,toc,page,header]{appendix} % REDONDANT VOIRE CONTRADICTOIRE AVEC these_main => en outre les options ne fonctionnent que dans un environnement appendices


\usepackage{wrapfig}
\usepackage[export]{adjustbox}
\usepackage{outlines}
\usepackage{graphicx}
\usepackage{pdfpages}
\usepackage{enumitem}
\usepackage{array}
\usepackage{tabularx}
\usepackage{here} %À ÉVITER, IL FAUT LAISSER LATEX PLACER LES FIGURES
\usepackage{libertine}
\usepackage{amsmath}
\usepackage{amssymb}
\usepackage{amsfonts}
\usepackage{mathrsfs} % POURQUOI NE PAS UTILISER \mathcal (standard) ?
\usepackage{pifont}
%\usepackage{nicefrac}		% INUTILE ?!
%\usepackage{mathtools}	        % INUTILE ?!
\usepackage[left,modulo,mathlines]{lineno} % après AMS + cf. patch ci-dessous
\usepackage[round,authoryear]{natbib}
\usepackage[utf8]{inputenc}
\usepackage[T1]{fontenc}
\usepackage[english,french]{babel}
\usepackage[bookmarks={true},bookmarksopen={true},pdfusetitle]{hyperref}
%\usepackage{footnotebackref} % NE FONCTIONNE PAS (?)
\usepackage{caption} % après babel, avant minitoc
\usepackage[francais]{minitoc} % après hyperref
\usepackage[nameinlink,noabbrev]{cleveref}
\usepackage[acronym,nomain,ucmark,automake]{glossaries}
\usepackage[framemethod=tikz]{mdframed}
\usepackage{tikz}
\usepackage{etoolbox} %pour patch lineno
%% PRÉAMBULE : PACKAGES fin



%% PRÉAMBULE : STYLE (début)

%% Format titre
\makeatletter
\renewcommand{\maketitle}{
  \bgroup
  \setlength{\parindent}{0pt}
  \begin{flushleft}
    {\Large \@title}
    \par\medskip
    \textbf{\@author}
    \par\medskip
    {\footnotesize \@date}
  \end{flushleft}
  \egroup
}


\makeatother

%% FORMAT PAGE
\geometry{left=3cm,top=2.5cm,bottom=2.5cm,headheight=0.7cm,headsep=1cm,footskip=1.25cm}
\addtolength{\skip\footins}{0.8\baselineskip} %extra skip before footnotes
\setlength{\footnotesep}{.5\baselineskip} %extra skip between footnotes
\usepackage[Bjornstrup]{fncychap}
\newcommand{\HRule}{\rule{\linewidth}{0.5mm}}

%% HEAD & FOOT
%% Pour faire plus "fancy", exemple ci-dessous :
%% - pages "plain" i.e. début chapitre, page de titre : juste numéro de
%% page au centre en bas ;
%% - pages normales, recto, en haut : numéro et titre section à gauche,
%% numéro de page à droite ; rien en bas ;
%% - pages normales, verso, en haut : numéro de page à gauche, numéro et
%% titre chapitre à droite ; rien en bas ;
%% numéros de page en gras, titres en oblique.
%%
\pagestyle{fancyplain}
%Head and foot definition:
%\position[\fancyplain{plain verso}{verso}]{\fancyplain{plain recto}{recto}}
\renewcommand{\chaptermark}[1]{\markboth{#1}{}} %left:chap+no# right:nothing
\renewcommand{\sectionmark}[1]{\markright{\thesection\ #1}} %right:sec+#
\lhead[\fancyplain{}{\bf\thepage}]{\fancyplain{}{\sl\rightmark}}
\rhead[\fancyplain{}{\sl\leftmark}]{\fancyplain{}{\bf\thepage}}
\cfoot[\fancyplain{\bf\thepage}{}]{\fancyplain{\bf\thepage}{}}
\chead{}\lfoot{}\rfoot{}
\addtolength{\headwidth}{7mm} %head expands on the outside

%% Pour que les chapitres commencent sur une page recto et que la page
%% verso blanche éventuelle soit de style plain ou vide :
\newcommand{\recto}{\newpage{\pagestyle{plain}\cleardoublepage}}
\newcommand{\rectovide}{\newpage{\pagestyle{empty}\cleardoublepage}}

\doublehyphendemerits=10000       % No consecutive line hyphens.
\brokenpenalty=10000              % No broken words across columns/pages.
\widowpenalty=9999                % Almost no widows at bottom of page.
\clubpenalty=9999                 % Almost no orphans at top of page.
\interfootnotelinepenalty=9999    % Almost never break footnotes.
\raggedbottom %Page non étirée pour placer la dernière ligne en bas de page


%% TABLES DES MATIÈRES
\setcounter{tocdepth}{3} %profondeur de la table des matières (3: -> subsubsection)
%\renewcommand{\contentsname}{Table des matières}
%\renewcommand{\listfigurename}{Liste des figures}
%\renewcommand{\listtablename}{Liste des tables}
%% Mini tables dans chaque chapitre
%\def\mtctitle{Contenu} %titre des minitoc
\setcounter{minitocdepth}{3} %profondeur de la minitoc (3: -> subsubsection)

%\usepackage{mtcoff} %annule proprement toutes les minitoc

%% ANNEXES
\AtBeginEnvironment{appendices}{\crefalias{section}{appendix}}
\Crefname{appendix}{l'Annexe}{les Annexes}
\Crefname{subappendix}{l'Annexe}{les Annexes}
\AtBeginDocument{%
  \def\subappendixautorefname{l'Annexe}
}

%% SECTIONS
\setcounter{secnumdepth}{3} %profondeur de numérotation (3: -> subsubsection)
%\renewcommand{\thechapter}{\Roman{chapter}} %nombres romains pour chapitres

%% SUBSUBSECTIONS
\renewcommand{\thesubsubsection}{\alph{subsubsection})} %lowercase letters

%% BIBLIOGRAPHIE
%\renewcommand{\bibname}{Bibliographie}

%% FLOATS = FIGURES & TABLES
\DeclareGraphicsExtensions{.pdf,.jpg,.png,.eps}
\graphicspath{{./Figures/PageTitre/}{./Figures/Chapitre1/}{./Figures/Chapitre2/}{./Figures/Chapitre3/}{./Figures/Chapitre4/}{./Figures/Chapitre5/}{./Figures/Annexes1/}}

%% Page fractions for floats
\renewcommand{\topfraction}{.7}       %default .7
\renewcommand{\bottomfraction}{.7}    %default .3
\renewcommand{\textfraction}{.2}      %default .2
\renewcommand{\floatpagefraction}{.5} %default .5
%% Distances between floats and text
\setlength{\floatsep}{2em}            %default 12pt float-float
\setlength{\textfloatsep}{\floatsep}  %default 18pt float[tb]-text
\setlength{\intextsep}{\floatsep}     %default \floatsep text-float-text

%\renewcommand{\tablename}{Tableau}
%\renewcommand{\figurename}{Figure}
%\def\figureautorefname{\figurename}
%\def\tableautorefname{\tablename}
\captionsetup{labelfont={bf}}

\newcolumntype{P}[1]{>{\centering\arraybackslash}p{#1}} % paragraphe centré
\renewcommand{\arraystretch}{1.5}

%% EQUATIONS
%\renewcommand{\theequation}{Eqn~\arabic{equation}}


\renewcommand{\theequation}{Eqn~\thechapter.\arabic{equation}}

%% COULEURS
\definecolor{navyblue}{rgb}{0.0, 0.0, 0.5}
\definecolor{shadecolor}{gray}{0.90}
\colorlet{mylinkcolor}{RoyalBlue}

%% LIENS
\hypersetup{colorlinks=true, allcolors=mylinkcolor}

%% GLOSSAIRES
\makeglossaries
\loadglsentries{Glossaire} %fichier externe
%% No colors for glossary links:
\renewcommand*{\glsdohyperlink}[2]{%
 {\hypersetup{linkcolor=RoyalBlue}\hyperlink{#1}{#2}}}
\glsenablehyper

%% PATCH pour compatibilité lineno & amsmath
%% Patch 'normal' math environment: 
\newcommand*\linenomathpatch[1]{%
  \expandafter\pretocmd\csname #1\endcsname {\linenomath}{}{}%
  \expandafter\pretocmd\csname #1*\endcsname{\linenomath}{}{}%
  \expandafter\apptocmd\csname end#1\endcsname {\endlinenomath}{}{}%
  \expandafter\apptocmd\csname end#1*\endcsname{\endlinenomath}{}{}%
}
%% Patch AMS math environment:
\newcommand*\linenomathpatchAMS[1]{%
  \expandafter\pretocmd\csname #1\endcsname {\linenomathAMS}{}{}%
  \expandafter\pretocmd\csname #1*\endcsname{\linenomathAMS}{}{}%
  \expandafter\apptocmd\csname end#1\endcsname {\endlinenomath}{}{}%
  \expandafter\apptocmd\csname end#1*\endcsname{\endlinenomath}{}{}%
}
%% Definition of \linenomathAMS depends on whether the mathlines option is provided
\expandafter\ifx\linenomath\linenomathWithnumbers
  \let\linenomathAMS\linenomathWithnumbers
  %% The following line gets rid of an extra line numbers at the bottom:
  \patchcmd\linenomathAMS{\advance\postdisplaypenalty\linenopenalty}{}{}{} 
\else
  \let\linenomathAMS\linenomathNonumbers
\fi
\linenomathpatch{equation}
\linenomathpatchAMS{gather}
\linenomathpatchAMS{multline}
\linenomathpatchAMS{align}
\linenomathpatchAMS{alignat}
\linenomathpatchAMS{flalign}


%% PRÉAMBULE : STYLE (fin)



%% ENVIRONNEMENTS & COMMANDES
%% à définir ci-dessous 

%% ENCADRE début
\newcommand{\titreencadre}{Titre}
\mdfsetup{skipabove=\topskip,skipbelow=\topskip}
\tikzset{titreblue/.style =
  {draw=RoyalBlue, line width=1.5pt, fill=white,
    rectangle, rounded corners, right,minimum height=2em}}
\makeatletter
\mdfdefinestyle{encadrestyle2}{%
  linewidth=1.5pt,roundcorner=5pt,linecolor=RoyalBlue,
  apptotikzsetting={\tikzset{mdfbackground/.append style ={%
        fill=gray!20}}},
  frametitlefont=\bfseries,
  singleextra={%
    \node[titreblue,xshift=2em] at (P-|O) %
    {~\mdf@frametitlefont{\titreencadre}\hbox{~}};},
  firstextra={%
    \node[titreblue,xshift=2em] at (P-|O) %
    {~\mdf@frametitlefont{\titreencadre}\hbox{~}};},
}
\newenvironment{encadre2}[1]{%
  %\nolinenumbers %patch ST pour éviter des sauts de pages mal placés
  \renewcommand{\titreencadre}{#1}
  \protected@edef\@currentlabelname{#1}
  \protected@edef\@currentlabel{#1}
  \begin{mdframed}[style=encadrestyle2,nobreak=true]
    \vspace{0.75\baselineskip}
    \begin{internallinenumbers}%patch ST pour numéroter les lignes des encadrés
    }{%
    \end{internallinenumbers}%patch ST
  \end{mdframed}
}
\makeatother
%% ENCADRE fin



%% Mise en forme du texte pour les unités        
%\usepackage{xspace}
\usepackage[load-configurations = abbreviations]{siunitx}
	\DeclareSIUnit{\MPa}{\mega\pascal}
	\DeclareSIUnit{\micron}{\micro\meter}
	\DeclareSIUnit{\tr}{tr}
	\DeclareSIPostPower\totheM{m}
	\sisetup{
   	locale = FR,
	  inter-unit-separator=$\cdot$,
	  range-phrase=~\`{a}~,     	% Utilise le tiret court pour dire "de... à"
	  range-units=single,  		% Cache l'unité sur la première borne
	  }




