
%%%%%%%%%%%%%%%%%%%%%%%%%%%%%%%%%%%%%%%%%%%%%%%%%%%%%%%%%%%%%%%%%%%%%%%%%%%%%%%%%%%%%%%%%%%%%
%%									Remerciements										  %
%%%%%%%%%%%%%%%%%%%%%%%%%%%%%%%%%%%%%%%%%%%%%%%%%%%%%%%%%%%%%%%%%%%%%%%%%%%%%%%%%%%%%%%%%%%%%


\chapter*{Remerciements}
Je tiens à remercier immensément  mes quatre encadrants de thèse Suzanne Touzeau
Caroline Djian-Caporalino, Vincent Calcagno, Ludovic Mailleret. Tout d’abord, pour m’avoir proposé ce sujet avec eux et pour m'avoir permis de découvrir le domaine passionnant de la recherche scientifique. Et en particulier à Suzanne Touzeau et Caroline Djian-Caporalino pour avoir  accepté de  diriger cette thèse en tant que mes deux co-directrices de  thèse. Je tiens à remercier tous mes encadrants  pour votre patience, vos conseils, votre bienveillance, votre disponibilité et vos nombreuses relectures de ce manuscrit. Merci
pour la qualité  de votre encadrement, de m'avoir fourni tout ce dont j'avais besoin (ordinateur, cluster de calcul, stages, écoles chercheurs, \textit{etc}...) qui m’ont permis d'effectuer cette thèse dans de très bonnes conditions. Les discussions que j'ai pu échanger  avec vous quatre ont toujours
été très constructives  et m’ont réellement aidé à avancer pour mener à bien ce projet. Je tiens  à souligner
votre soutien considérable aussi bien scientifique et personnel  grâce auquel j’ai pu surmonter les moments parfois plus difficiles.

Plus spécifiquement, j’adresse mes remerciements à ma directrice de thèse, Suzanne Touzeau, pour son soutien et  toute son immense implication tout au long de cette thèse.  Je lui exprime infiniment toute ma gratitude pour ses
précieux conseils, sa prévenance avec moi  ses qualités humaines et professionnelles exceptionnelles.
Je tiens également à remercier, Ludovic Mailleret, responsable de l’équipe M2P2, pour son dévouement, son investissement et ses conseils pertinents qui m'ont poussé à m'améliorer.
Je tiens à remercier Vincent Calcagno pour son brillant enseignement, sa disponibilité, ses conseils précieux et toujours pertinents.  
Je remercie également Caroline Djian-Caporalino, ma co-directrice de thèse, qui m’a apporté un appui considérable et remarquable. Je la remercie pour son appui dans la conception de la partie expérimentale de cette thèse et pour l'accès à de nombreuses données expérimentales. Et également,  pour le partage d'une large  bibliographie sur la biologie des nématodes qui m'a été d'une aide  extrêmement précieuse pour  la rédaction de ce manuscrit.
Je remercie Nathalie Marteu pour ses conseils, son aide précieuse et son implication dans les expérimentations d'infection de tomates par les nématodes, sans qui cette partie expérimentale de ma thèse n’aurait pas pu se faire.

Je souhaite également exprimer  mes sincères reconnaissances  à  Ivan Sache et Gaël Thébaud pour avoir accepté d'être rapporteur de ma thèse et d’évaluer mon travail. Je remercie également Pierre Abad, Florence Carpentier d’avoir accepté d'être examinateur de ma thèse et de participer à mon jury de thèse. Dans le cadre de cette thèse inter-disciplinaire, je tiens à souligner l’investissement de chacun pour la compréhension de mon travail et de son évaluation.

Je suis particulièrement reconnaissant des nombreuses opportunités de rencontres qui m’ont
été fournies durant cette thèse. Tout d’abord, à travers le choix  expérimenté et réfléchi de l’ensemble
des membres de mon comité de thèse : 
 Pierre Abad,
 Frederic Hamelin,
 Frederic Fabre,
 Florence Carpentier,
 Bernard Caromel,
 Benoit Borschinger ;
que je souhaite remercier pour m’avoir permis d'avancer. J’ai pu bénéficier d’un suivi multi-disciplinaire très constructif et de qualité.
Ensuite, via les différentes écoles-chercheurs  auxquelles j’ai pu participer. 
J’ai eu  le plaisir de participer à l’école chercheur « Analyse de sensibilité, métamodélisation et
optimisation de modèles complexes » (MEXICO) qui s’est déroulée à la Rochelle% du 26 au 30 mars 2018. 
. Je remercie tous les intervenants pour la qualité de leur enseignement et pour les échanges très enrichissants. Je remercie également le comité d'organisation (Robert Faivre, Julien Bect, Victor Picheny, Bastien Roux), pour avoir mener à bien cette école chercheur, pour la bonne ambiance de travail et pour les occasions de convivialité  avec les autres participants. 
 j’ai également eu le plaisir de suivre  l'école d'été « Mathématiques sur l'écologie et l'évolution » en Finlande,
organisée par le groupe Biomathématiques de l'Université d'Helsinki. %du 19 au 26 août 2018.
 Je remercie là encore l’ensemble des intervenants et des organisateurs qui ont grandement participé
à la qualité de cette formation, ainsi qu’à son ambiance 
plaisante  et conviviale! % J'ai également une pensée amicale pour tous les participants, avec qui j'ai passé de très bons moments.22-24 Mars 2017
Enfin, via les différentes conférences  auxquelles j’ai assisté et l'enseignement que j'ai pu effectuer. Une de mes toutes premières expériences fut lors  d'une conférence internationale pour jeunes chercheurs (Ecology and Agriculture
Summit for Young scientists, CEBC Chizé), où j’ai eu le plaisir
d’échanger avec Thomas Perrot dont ma thèse fait suite à ses travaux et ceux de Mathilde Mercat. J'ai pu participer à l' ECMTB (European Conference on Mathematical and Theoretical Biology) durant laquelle j’ai pu retomber par hasard sur un de mes professeurs de Master Laurent Pujo-Menjouet,  ce qui fut pour moi une très bonne surprise. Je remercie également Suzanne Touzeau pour m'avoir permis à nouveau d'effectuer une expérience enrichissante dans l'enseignement en tant que chargés de travaux dirigés en analyse de données à Polytech Nice Sophia.% Je remercie également Armelle Favery  et Véronique Oiknine du service Communication Inrae pour  m'avoir donné l'opportunité d'animer des ateliers pour des élèves de collèges  dans le cadre de la journée
%international « Fascination of plants Day »  pour sensibiliser les plus jeunes à l’environnement. Je remercie également  Mme  pour m'avoir permis de présenter mes travaux lors de la 2éme journée de l’environnement au lycée Saint Joseph à Carnoles, sur le thème : se nourrir.

Je tiens également à souligner mon plaisir à avoir effectué cette thèse au sein des instituts d’accueil  ISA (Institut Sophia Agrobiotech)  et INRIA. Je remercie  les directeurs d’unité de l'ISA successifs Pierre Abad et Philippe
Castagnone   pour leurs conseils précieux et la qualité de nos  échanges très constructifs. 
Je remercie également l'ensemble du personnel de l'unité ISA pour leur accueil et les bons moments passés durant ma thèse. Je souhaite aussi adresser particulièrement un immense merci aux membres des équipes M2P2, IPN de l'ISA et BIOCORE de l'INRIA pour leur accueil chaleureux,  leur gentillesse à mon égard et la bonne ambiance durant ma thèse!  Je vais certainement oublier des personnes dans cette liste. Je m’excuse donc par avance pour ces
oublis, mais je tiens tout de même à remercier Cécile, Lydia, Léo, Marine, Hisham, Guy, Bruno, Christine, Séverine, Louise, Valentina de l'équipe M2P2  pour l'ambiance de convivialité et tous les bons moments de partages.
J'ai une pensée  pour tous les stagiaires, doctorants, post-doctorants et ingénieurs que j’ai côtoyés pendant ma thèse, avec qui ces années sont devenues inoubliables, et qui sans eux  cette expérience aurait été tout autre! Je souhaiterais remercier les anciens stagiaires Arthur, Hugo, Robin, Rozenn, Lisa, David, Thomas, Baptiste, Faten, Lucas, Aude  et  doctorants Victor  de l'équipe M2P2 pour leur bonne humeur, les moments partagés et d’entraides. Clin d’œil pour Flora, ce fut un réel plaisir de partager le bureau avec toi au cours de ma thèse, ta joie de vivre et les moments en ta compagnie vont me manquer. Je te souhaite le meilleur pour la fin de ta thèse et la suite. J'adresse également mes remerciements aux doctorants et post doctorants Lucie, Marjorie, Walid, Carlos, Israël de l’équipe Biocore, Lucie, Geoffrey, Danila, Laïla, Camille, Silène, Salma, Dries, Fatima et Michela des différentes équipes  de l'ISA pour tous ces bons moments de partages toujours dans la bonne humeur!


Je souhaiterais terminer ces remerciements avec une pensée  à ma famille : mes parents, mes sœurs Laura, Marie-jo et Karine, et mes amis  qui ont toujours été là pour moi et ont su m’accompagner  tout au long de cette aventure.